\documentclass{beamer}
\usepackage[T1, T2A]{fontenc}
\usepackage[utf8]{inputenc}	
\usepackage[english,ngerman]{babel} 
% Kyrillisch und Deutsche Zeichensatz
\usepackage{colortbl}
\usepackage{eurosym}
\usepackage{beamerthemesplit}
\begin{document}
\title{Einführung in Docker} 
\author{Jörg Reuter}
\date{\today} 

\frame{\titlepage} 

\frame{\frametitle{Docker - Alter Wein in neuen Schläuchen?} 
    Gibt es schon lange:
    \begin{itemize}
        \item Containe (OpenVZ, Solris Zones, LXC)r
        \item chroot jail
    \end{itemize}
    und hatten alle Ihre Nachteile ...
}

\frame{\frametitle{Vorteile von Docker} 
    \begin{itemize}
        \item Docker ist sehr "leichtgewichtig" z.B. 2500 WebServer unter Docker auf einem Rasberry Pi 2: \url{https://blog.docker.com/2015/10/raspberry-pi-dockercon-challenge-winner/})
        \item Docker verwendet modernste Kernel-Feauture wie:
        \begin{itemize}
            \item eigenes Netzwerk.
            \item Eigene Gruppen und Namespaces.
            \item Eigener Speicher.
            \item Eigenes Ressourcenmangment.
            \item Extrem einfacher Zugriff auf Container. 
            \item Container können als Textdatei weitergeben werden.
        \end{itemize}
    \end{itemize}    
}

\frame{\frametitle{Docker besteht aus fünf Teilen:}
    \begin{itemize}
        \item Docker Client
        \item Docker Images  
        \item Register
        \item Docker Container
        \item Docker Dämon/Server
    \end{itemize} 
    Aufbau: \url{https://docs.docker.com/engine/installation/images/win_docker_host.svg}
}

\frame{\frametitle{An die Tastatur}
    Herunter laden der notwendigen Schlüssel und Präsentation:
    \begin{itemize}
        \item Schlüssel und Unterlagen: \url{https://github.com/joergre/workshops.git}
        \item Essenbestellung
        \item Putty für Windows: \url{http://www.putty.org} 
        \item iTerm2 für Mac: \url{https://www.iterm2.com}
    \end{itemize} 
}

\frame{\frametitle{Websites}
    \begin{itemize}
        \tem test
    \end{itemize}    
}

\end{document}
