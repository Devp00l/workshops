% Created 2016-04-19 Tue 10:01
\documentclass[11pt]{article}
\usepackage[utf8]{inputenc}
\usepackage[T1]{fontenc}
\usepackage{fixltx2e}
\usepackage{graphicx}
\usepackage{grffile}
\usepackage{longtable}
\usepackage{wrapfig}
\usepackage{rotating}
\usepackage[normalem]{ulem}
\usepackage{amsmath}
\usepackage{textcomp}
\usepackage{amssymb}
\usepackage{capt-of}
\usepackage{hyperref}
\author{Jörg Reuter}
\date{}
\title{Regression Models in R}
\hypersetup{
 pdfauthor={Jörg Reuter},
 pdftitle={Regression Models in R},
 pdfkeywords={},
 pdfsubject={},
 pdfcreator={Emacs 24.5.1 (Org mode 8.3.4)}, 
 pdflang={English}}
\begin{document}

\maketitle
\tableofcontents


\section*{Introduction}
\label{sec:orgheadline2}

\subsection*{Workshop description}
\label{sec:orgheadline1}

\begin{itemize}
\item This is an intermediate/advanced R course
\item Appropriate for those with basic knowledge of R
\item This is not a statistics course!
\item Learning objectives:
\begin{itemize}
\item Learn the R formula interface
\item Specify factor contrasts to test specific hypotheses
\item Perform model comparisons
\item Run and interpret variety of regression models in R
\end{itemize}
\end{itemize}


\section*{Einführung in Docker}
\label{sec:orgheadline3}
Wir laden das Image von Ubuntu 14.04 herunter:

\#+begin\(_{\text{src}}\) bash
docker pull ubuntu:14.04
\#+end\(_{\text{src}}\) bash

Anzeige der lokal vorhanden Images:

\#+begin\(_{\text{src}}\) bash
docker images
\#+end\(_{\text{src}}\) bash

Zugriff auf die Shell

\#+begin\(_{\text{src}}\) bash
docker run -t -i ubuntu:14.04 bash
\#+end\(_{\text{src}}\) bash

Wir benötigen jetzt eine zweite Konsole die nicht im Docker-Container angemeldet ist. Diese Konsole wird benötigt um den Docker-Container zu verwalten.
Von einer zweiten Konsole:
\#+begin\(_{\text{src}}\) bash
docker ps
\#+end\(_{\text{src}}\) bash

um sich den Namen des laufenden Docker Container anzeigen zu lassen 
Docker Container anhalten (hungy\(_{\text{euklid}}\) ist der Name der jeweiligen Maschine):

\#+begin\(_{\text{src}}\) bash
docker stop hungry\(_{\text{euclid}}\)
\#+end\(_{\text{src}}\) bash

Gestoppten Container neu starten:

\#+begin\(_{\text{src}}\) bash
docker start -i hungry\(_{\text{euclid}}\)
\#+end\(_{\text{src}}\) bash

Container löschen:

\#+begin\(_{\text{src}}\) bash
docker stop hungry\(_{\text{euclid}}\)
docker rm hungry\(_{\text{euclid}}\)
\#+end\(_{\text{src}}\) bash

Nginx Image herunterladen, Conntainer starten und den Namen nginx geben:
docker run -d --name nginx nginx

\#+begin\(_{\text{src}}\) bash
docker ps -s
\#+end\(_{\text{src}}\) bash

Alle laufenden und gestoppten Container löschen:

\#+begin\(_{\text{src}}\) bash
docker ps -qa|xargs docker rm -f
\#+end\(_{\text{src}}\) bash

\section*{Zum Schluss}
\label{sec:orgheadline7}

\subsection*{Helfe mir, den Kurs besser zu machen!}
\label{sec:orgheadline4}
\begin{itemize}
\item Bitte nehme Dir einen Moment Zeit und fülle das Feedback-Formular aus.
\item Der Kurs existiert für Dich -- sage mir, was Du brauchst!
\item * Wrap-up
\end{itemize}

\subsection*{Help us make this workshop better!}
\label{sec:orgheadline5}
\begin{itemize}
\item Please take a moment to fill out a very short
\end{itemize}
feedback form 
\begin{itemize}
\item These workshops exist for you -- tell us what you need!
\item \url{http://goo.gl/forms/04cJw2mtBB}
\end{itemize}

\subsection*{Weitere Informationsquellen}
\label{sec:orgheadline6}
\end{document}