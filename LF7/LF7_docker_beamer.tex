\documentclass{beamer}
\usepackage[T1, T2A]{fontenc}
\usepackage[utf8]{inputenc}	
\usepackage[english,ngerman]{babel} 
% Kyrillisch und Deutsche Zeichensatz
\usepackage{colortbl}
\usepackage{eurosym}
\usepackage{beamerthemesplit}
\begin{document}
\title{Einführung in Docker} 
\author{Jörg Reuter}
\date{\today} 

\frame{\titlepage} 

\frame{\frametitle{Inhaltsverzeichnis}\tableofcontents} 

\section{Einführung} 
\frame{\frametitle{Docker - Alter Wein in neuen Schläuchen?} 
Gib te schon lange:
\begin{itemize}
\item Containe (OpenVZ, Solris Zones, LXC)r
\item chroot jail
\end{itemize}
und hatten alle Ihre Nachteile ...
    
}
\subsection{Warum Docker}
\frame{ 
\begin{itemize}
    \item Docker ist sehr "leichtgewichtig" z.B. 2500 WebServer unter Docker auf einem Rasberry Pi 2: \url{https://blog.docker.com/2015/10/raspberry-pi-dockercon-challenge-winner/})
    \item Docker verwendet modernste Kernel-Feauture wie:
    \begin{itemize}
        \item eigenes Netzwerk.
        \item Eigene Gruppen und Namespaces.
        \item Eigener Speicher.
        \item Eigenes Ressourcenmangment.
        \item Extrem einfacher Zugriff auf Container. 
        \item Container können als Textdatei weitergeben werden.
    \end{itemize}
\end{itemize}    
}


\section{Komponenten von Docker} 
\frame{\frametitle{Docker besteht aus fünf Teilen:}
\begin{itemize}
    \item Docker Client
    \item Docker Images  
    \item Register
    \item Docker Container
    \item Docker Dämon/Server
\end{itemize} 
Aufbau: \url{https://docs.docker.com/engine/installation/images/win_docker_host.svg}
}

\frame{\frametitle{An die Tastatur}
Herunter laden der notwendigen Schlüssel und Präsentation:
\begin{itemize}
    \item Schlüssel und Unterlagen: \url{https://github.com/joergre/workshops.git} \pause
    \item Essenbestellung\pause
    \item Putty für Windows: \url{http://www.putty.org} \pause 
    \item iTerm2 für Mac: \url{https://www.iterm2.com/}\pause
\end{itemize} 
}

\section{Hier wird Dir gehlofen}
\frame{\frametitle{Websites}
\begin{itemize}
    \item Docker Homepage (\url{https://www.docker.com})\pause
    \item Docker Hub (\url{https://hub.docker.com}) \pause
    \item Docker Blog (\url{https://blog.docker.com}) \pause
    \item Docker Dukumentation ({\url{https://docs.docker.com})\pause
    \item Docker "Getting Started Guide" (\url{https://docs.docker.com/mac/})\pause
    \item Docker Code - Github (\url{https://github.com/docker/docker})\pause
    \item Docker Forge -  Sammulung von Tools (\url{https://github.com/dockerforge}) \pause
\end{itemize}
}


\frame{\frametitle{Sonstige Hilfen im Netz}
\begin{itemize}
    \item Docker Mailing-Liste (\url{https://groups.google.com/forum/#!forum/docker-user})\pause 
    \item Docker auf IRC (irc.freenode.net Kanal: #docker) \pause 
    \item Docker auf Twitter (\url{https://twitter.com/docker})\pause 
    \item Docker auf StackOverflow (\url{https://stockoverflow.com/search?q=docker})
    \item Homepage von Docker: (\url{https://docker.com})
\end{itemize}
}

\section{Installation} 
\frame{\frametitle{Targets}
\begin{itemize}
    \item Alles mit Linux
    \item BSD experimentell, läuft aber problemlos
    \item Windows Server ab 2016
    \item Mac und Windows (ab 7.1): Docker Toolbox
    \item Amazon EC2
    \item Racksppace Cloud
    \item Google Compute Engine
    \item ...
\end{itemize}
}
\subsection{Einschränkung Targets}
\frame{\frametitle{Einschränkungen}
\begin{itemize}
    \item 64-bit
    \item Linux-Kernel ab 3.8
\end{itemize}

\section{Installation von Docker}
\frame{\frametitle{Befehle - Voraussetzung}



\end{document}
